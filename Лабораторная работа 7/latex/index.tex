{\bfseries{ Лабораторна робота № 7. Поліморфізм }} ~\newline
{\bfseries{Мета роботи\+:}}\+: отримати знання про парадигму ООП – поліморфізм; навчитися застосовувати отримані знання на практиці. ~\newline
 {\bfseries{Індивідуальне завдання\+:}} Модернізувати попередню лабораторну роботу шляхом\+:
\begin{DoxyItemize}
\item додавання ще одного класу-\/спадкоємця до базового класу. Поля обрати самостійно;
\item базовий клас зробити абстрактним. Додати абстрактні поля;
\item розроблені класи-\/списки поєднуються до одного класу таким чином, щоб він міг працювати як з базовим класом, так і з його спадкоємцями. При цьому серед полів класу-\/списку повинен бути лише один масив, що містить усі типи класів ієрархії. Оновити методи, що працюють з цим масивом ~\newline
 \begin{DoxyAuthor}{Author}
Богданов І. 
\end{DoxyAuthor}
\begin{DoxyDate}{Date}
04.\+04.\+2020 
\end{DoxyDate}
\begin{DoxyVersion}{Version}
1.\+0 
\end{DoxyVersion}

\end{DoxyItemize}